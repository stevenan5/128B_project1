\documentclass[12pt]{article}
 
\usepackage[margin=1in]{geometry}
\usepackage{amsmath,amsthm,amssymb}
\usepackage{mathtools}
\newcommand{\N}{\mathbb{N}}
\newcommand{\R}{\mathbb{R}}
\newcommand{\Z}{\mathbb{Z}}
\newcommand{\Q}{\mathbb{Q}}

\newenvironment{problem}[2][Problem]{\begin{trivlist}
\item[\hskip \labelsep {\bfseries #1}\hskip \labelsep {\bfseries #2}]}{\end{trivlist}}
\newenvironment{exercise}[2][Exercise]{\begin{trivlist}
\item[\hskip \labelsep {\small\bfseries #1}\hskip \labelsep {\small\bfseries #2}]}{\end{trivlist}}
\newenvironment{epart}[2][Part]{\begin{trivlist}
\item[\hskip \labelsep {\footnotesize\bfseries #1}\hskip \labelsep {\footnotesize\bfseries #2}]}{\end{trivlist}}

\DeclarePairedDelimiter\abs{\lvert}{\rvert}
\DeclarePairedDelimiter\norm{\lVert}{\rVert}

\makeatletter
\let\oldabs\abs
\def\abs{\@ifstar{\oldabs}{\oldabs*}}

\let\oldnorm\norm
\def\norm{\@ifstar{\oldnorm}{\oldnorm*}}
\makeatother

\DeclarePairedDelimiter\ceil{\lceil}{\rceil}
\DeclarePairedDelimiter\floor{\lfloor}{\rfloor}

\newtheorem{theorem}{Theorem}[section]
\newtheorem{corollary}{Corollary}[theorem]
\newtheorem{lemma}[theorem]{Lemma}
 
\begin{document}
 
\title{Project 1}
\author{Steven An, Kyle Cox, Sangwon Yoon}
\date{Feburary 21, 2018}
\maketitle

\subsection*{Distribution of work}
Although Steven did most of the coding, the writing of the pseudo code was done by all members and sent to him to implement.
The other two group members checked on his work as he implemented it.
The following page includes a screenshot of the commit history on github.

\subsection*{Part (i)}
See attached MatLab document.
The code from p. 100 is implemented with the \textit{for} loop being changed to \textit{parfor} and the inline function to anonymous function to save computation time.

\subsection*{Part (ii)}
When $\abs{z} > 2$, then the point is considered to diverge and the initial value in the orbit is assigned a value accordingly.
Changing the initial values, $z_0$, results in showing either more or less of the fractal.
In the code attached, the range of initial values was increased so that the full fractal can be seen.

\subsection*{Part (iii)}
Program implemented as detailed.
The Julia sets that are the results match the filled Julia sets computed in (ii).

\subsection*{Part (iv)}

\subsection*{Part (v)}
See attached.

\subsection*{Part (vi)}
The fixed point computation is vectorized for computation savings.
While the value $|z_k|$ in the iteration is less than 100, the iteration count increments by one.
The image is generated by assigning blue values to small iteration count for divergence with a gradation to red for high divergence iteration counts.
Finally, the values that never diverge are colored black.

\subsection*{Part (vii)}
Here, the areas that converge to certain roots of unity, the basins, are color coded.
It can be seen that where the trails converge to in the picture are the roots of unity.
The roots of unity are marked with asterisks to show that the computation was correct.
The redder the color, the faster it was to converge to a root.
As the color changes its shade to blue, the iterations required 
to converge increases.

\subsection*{Part (viii)}
The coloring scheme is the same as the last section.
The whole Mandelbrot set is graphed, and a section, that was our best guess as to what was shown on the 1985 Scientific American cover is also shown.

\end{document}