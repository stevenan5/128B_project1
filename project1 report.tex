\documentclass[12pt]{article}
 
\usepackage[margin=1in]{geometry}
\usepackage{amsmath,amsthm,amssymb}
\usepackage{mathtools}
\newcommand{\N}{\mathbb{N}}
\newcommand{\R}{\mathbb{R}}
\newcommand{\Z}{\mathbb{Z}}
\newcommand{\Q}{\mathbb{Q}}

\newenvironment{problem}[2][Problem]{\begin{trivlist}
\item[\hskip \labelsep {\bfseries #1}\hskip \labelsep {\bfseries #2}]}{\end{trivlist}}
\newenvironment{exercise}[2][Exercise]{\begin{trivlist}
\item[\hskip \labelsep {\small\bfseries #1}\hskip \labelsep {\small\bfseries #2}]}{\end{trivlist}}
\newenvironment{epart}[2][Part]{\begin{trivlist}
\item[\hskip \labelsep {\footnotesize\bfseries #1}\hskip \labelsep {\footnotesize\bfseries #2}]}{\end{trivlist}}

\DeclarePairedDelimiter\abs{\lvert}{\rvert}
\DeclarePairedDelimiter\norm{\lVert}{\rVert}

\makeatletter
\let\oldabs\abs
\def\abs{\@ifstar{\oldabs}{\oldabs*}}

\let\oldnorm\norm
\def\norm{\@ifstar{\oldnorm}{\oldnorm*}}
\makeatother

\DeclarePairedDelimiter\ceil{\lceil}{\rceil}
\DeclarePairedDelimiter\floor{\lfloor}{\rfloor}

\newtheorem{theorem}{Theorem}[section]
\newtheorem{corollary}{Corollary}[theorem]
\newtheorem{lemma}[theorem]{Lemma}
 
\begin{document}
 
\title{Project 1}
\author{Steven An, Kyle Cox, Sangwon Yoon}
\date{Feburary 21, 2018}
\maketitle

\subsection*{Distribution of work}
Even though most of the commits are by Steven, he was the one who put the code on his computer to test as he had the fastest computer.
Case in point is where one member ran the whole program for 45 minutes to no avail while finished in under 11 minutes on Steven's computer.
The pseudo code was written as a group effort and then handed off.
All members of the group contributed to the debugging effort in person.


\subsection*{Part (i)}
See attached MatLab document.
The code from p. 100 is implemented with the \textit{parfor} instead of the \textit{for} and the inline function to anonymous function to save computation time.
Computation time for the first figure went from 1 minute 10 seconds to less than 5 seconds.

\subsection*{Part (ii)}
When $\abs{z} > 2$, then the point is considered to diverge and the initial value in the orbit is not changed from 2 (to represent white in the custom colormap).
Changing the initial values, $z_0$, results in showing either more or less of the fractal.
In the code attached, the range of initial values was increased so that the full fractal can be seen.

\subsection*{Part (iii)}
Program implemented as detailed.
The Julia sets that are the results match the filled Julia sets computed in (ii).

\subsection*{Part (iv)}

\subsection*{Part (v)}
See attached.
For the four values of $c$ chosen, the computer showed that their respective Julia sets were connected because $orb(0)$ did not diverge.

\subsection*{Part (vi)}
The fixed point computation is vectorized via meshgrid for computation savings.
While the value $|z_k|$ in the iteration is less than 100, the iteration count increments by one.
The image is generated by assigning blue values to small iteration count for divergence with a gradation to red for high divergence iteration counts.
Finally, the values that never diverge are colored black.

\subsection*{Part (vii)}
Here, the areas that converge to certain roots of unity, the basins, are color coded.
It can be seen that where the trails converge to in the picture are the roots of unity.
They are marked with asterisks to show that the computation was correct.
The redder (and darker) the color, the faster it was to converge to a root.
As the color changes its shade to blue, the iterations required 
to converge increases.

\subsection*{Part (viii)}
The coloring style is the same as the last section. 
A custom colormap was created from $jet()$ to try and emphasize the border of the set more.
The whole Mandelbrot set is graphed, and a section, that was our best guess as to what was shown on the 1985 Scientific American cover, is also shown.

\end{document}